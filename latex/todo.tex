\documentclass[12pt]{article}
\usepackage{amsmath, amssymb, geometry, graphicx}
\geometry{a4paper, margin=1in}

\title{Assessing Group Dominance: A Novel Method for Ranking and Analysis}
\author{Kartikeya Mishra}
\date{\today}

\begin{document}
	
	\maketitle
	
	\begin{abstract}
		This paper introduces a novel method for calculating group-level dominance scores, incorporating rank sums, weight bias adjustments for unequal group sizes, and the concept of the Upper Dominant Half (UDH). This method offers a fair and scalable approach for comparing group performance, addressing limitations in existing dominance ranking methods. Empirical validation and theoretical implications are discussed.
	\end{abstract}
		
		\section{Introduction}
		- Overview of dominance ranking methods in psychology and related fields.
		- Limitations of existing methods (e.g., David's Score, Elo-Rating).
		- Purpose of this research: Introducing a group-level dominance calculation method.
		
		\section{Proposed Methodology}
		\subsection{Key Components}
		\begin{itemize}
			\item \textbf{Ranks}: Sequential ranks assigned to items within and across groups. Where there is \[k = \text{number of groups}, \quad n_i = \text{number of items in each group }\]
			\begin{align}
				\text{N} = \sum\limits_i^k n_i, \quad  n_i = \text{Number of items in each group}
			\end{align}
			\item \textbf{Upper Dominant Half (UDH)}:
			\begin{align}
				\text{UDH} = \frac{N(N+1)}{2} - \frac{(a-1)a}{2}, \quad a = \lceil N/2 \rceil
			\end{align}
			\item \textbf{Weight Bias ($w_i$)}:
			\begin{align}
				w_i = 1 - \frac{n_i - X_{EI}}{n_i}, \quad X_{EI} = \frac{N}{k}
			\end{align}
		\end{itemize}
		
		\subsection{Dominance Score Formula}
		\begin{align}
			U_i = w_i \cdot \frac{\sum R_i}{\text{UDH}}
		\end{align}
		\begin{itemize}
			\item $\sum R_i$: Sum of ranks for group $i$.
			\item $UDH$: Benchmark for dominance potential.
			\item $w_i$: Adjustment for group size bias.
		\end{itemize}
		
		\section{Empirical Validation}
		\subsection{Simulated Examples}
		- Description of simulations with varied group sizes and ranks.
		- Verification of theoretical bounds ($U_i \leq 1$).
		
		\subsection{Case Studies}
		- Real-world examples demonstrating the method's application.
		- Comparison with existing methods (e.g., David's Score).
		
		\section{Discussion}
		\subsection{Theoretical Implications}
		- Fairness in dominance ranking.
		- Applicability to group-level analyses.
		
		\subsection{Practical Applications}
		- Group comparisons in psychological experiments.
		- Applicability to interdisciplinary fields (e.g., organizational behavior, education).
		
		\section{ToDo}
		\begin{itemize}
			\item remember to mention this is better for non-parametric ordinal data etc. Test a parametric significance level with the data and compare the significance to prove that it is not suitable for normal distribution etc. data
			\item prove the range of the normalized rank (is it 0 to 1, or 0 to infinity)
			\item Compare different research methods (Kruskall wallis 'H' test already an extension of Man whitney U test)
			\item summary of findings
			\item visual intuition of the derivation
			\item get an dominance example (3 number of groups, unequal items in group, tied ranks etc., odd total number of groups for ceiling)
			\item annotate for bibliography
                \item Overview of dominance ranking methods in psychology and related fields.
    \item Limitations of existing methods (e.g., David's Score, Elo-Rating).
    \item Purpose of this research: Introducing a group-level dominance calculation method.
		\end{itemize}
		
		
		\section{Conclusion}
		
		
		
		\section*{References}
		- Placeholder for references to key papers and prior work.
		
	\end{document}
